\documentclass[12pt]{article}
\usepackage[english]{babel}
\usepackage[utf8x]{inputenc}
\usepackage[T1]{fontenc}
\usepackage{scribe}
\usepackage{listings}
\usepackage{framed}

\Scribe{Team ID - 24\footnotemark}
\footnotetext[1]{Team members: 23m2154, 23m2156, 23m2157, 23m2158, 23m2162, 23d1596}
\Lecturer{Sunita Sarawagi}
\LectureNumber{1 - 4}
\LectureDate{November 16, 2023}
\LectureTitle{Practice Questions}

\lstset{style=mystyle}

\begin{document}
	\MakeScribeTop
	
	%#############################################################
	%#############################################################
	%#############################################################
	%#############################################################
	
	\section*{Question 1: Linear Algebra}
	Suppose $x_k$ is the fraction of IIT Bombay students who prefer Foundations of Machine Learning (FML) over Advanced Machine Learning (AML) at year k. The remaining fraction $y_k = 1-x_k$ prefers AML. At year $k+1$, $\frac{1}{5}$ of those who prefer FML changed their mind (possibly after taking EE 768). Also at year $k+1$, $\frac{1}{10}$ of those who prefer AML change their mind (possibly because of the exams!). Create the matrix $A$ to give $[x_{k+1} \quad y_{k+1}]^T = [x_k \quad y_k]^TA$ and find the limit of $[1 \quad 0]A^T$ as $k \rightarrow \infty$.
	
	\begin{framed}
		\textbf{Solution:}
		\begin{equation}
			A = \begin{bmatrix}
				0.8 & 0.1 \\
				0.2 & 0.9 \\
			\end{bmatrix}
		\end{equation}
		
		The eigenvector with $\lambda = 1$ is $[0.333 \quad 0.667]^T$. This is the steady state starting from $[1 \quad 0]^T$. $\frac{2}{3}$ of all students prefer AML!
	\end{framed}
	
	\section*{Question 2: Linear Regression}
	Suppose we are in the year of 2026. The LIBS instrument of India's Chandrayaan-4 rover has made the first-ever in-situ measurements on the search of water ice beneath the lunar surface near the south pole of Moon. The instrument has just sent back exciting data giving the concentration of water ice y at depth x beneath the surface of the south pole on Moon!

	Your task, as one of the mission specialists (back on Earth), is to figure out what hypothesis best models the data, which is shown in Figure \ref{F:1}:

	\myfig{.5}{figure1.png}{The concentration of water ice y at depth x beneath the surface of the south pole on Moon}{F:1}	
	
	Assume that the datapoints shown in this figure come from three disjoint subsets:
	\begin{enumerate}
		\item[A:] Depth $x = 0$ to $x=6$ (circles)
		\item[B:] Depth $x = 6$ to $x =12$ (squares)
		\item[C:] Depth $x>12$ (the symbol ×)
	\end{enumerate}

	And as an ML expert, you know that while you may train your model on one subset of data, you should test it on a different subset of data.
	
	a) Suppose your hypothesis is that ice concentration is linearly related to depth, i.e. $y = \theta^T x + \theta_0$. You employ mean square error (MSE) for the objective function, and use dataset A for training, and dataset B for testing (since they are conveniently disjoint!). Let us say that that MSE below 30 is LOW, and MSE above 100 is HIGH. Judging from the above plot, will the MSE for training be LOW or HIGH? How about for testing? Explain why.
	
	b) Continuing with the hypothesis that ice concentration is linearly related to depth, you now employ datasets A and B (combined) for training, and dataset C for testing. Judging from the above plot, will the MSE for training be LOW or HIGH? How about for testing? Are your choices for training and testing datasets good ones? Explain.
	
	c) Realizing that Moon is unlikely to be a snowball of ice (although it’s possible Earth once was!), you switch to a family of hypotheses with nonlinear feature transforms, $y = \theta^T \phi_k(x) + \theta_0$, where $\phi_k(x)$ is a vector of polynomials up to order $k$. Can you think of any good way to evaluate what order $k$ is the best to choose? Explain.
	
	\vspace{10pt}
	
	\begin{framed}
		\textbf{Solution: }
		a) Training Error: LOW. Testing Error: LOW.\\
		Both errors are LOW because training on dataset A should produce a straight line which fits both A and B very well. 
		
		b) Training Error: LOW. Testing Error: HIGH. \\
		Training error will be LOW because training on dataset A and B should produce a straight line which fits both A and B very well. However, extrapolating forward the straight line produced will not be a good fit for dataset C leading to a HIGH testing MSE.
		
	 	If we are trying to model all of the data (i.e. the data in subsets A, B, and C), the union of subsets A and B is not representative; it misses out on the behavior in subset C. Similarly, subset C is not representative; it misses out on the behavior in subsets A\&B. A better choice of training data would be one that has points from every subset; similarly, a better choice of testing data would have points from every subset.
		
		c) Training Set: randomly select data points from across all three datasets (A, B, C). A good percentage could be 80\% data for training.\\
		Testing Set: use the remaining 20\% points not chosen for training to be part of the test set.
		
		The reason one would want to choose randomly from across all datasets is because the data for training and for testing should come from the same sample distribution, even if they are disjoint datapoints. Alternatively, use cross-validation. With cross-validation, you could use all the data for training then determine the best $k$ by minimizing the error output by cross-validation. This would mean no need for a single separate test set.
	\end{framed}

	\section*{Question 3: Logistic Regression}
	For the binary logistic regression problem, the target values are encoded as $t^i \in \{0,+1\}$.
	For a dataset $D_N = \{(x^{(i)}, t^{(i)})\}$ with $t^i \in \{0,+1\}$, the logistic regression is defined using the following steps:
	\begin{align}
		z &= w^Tx + b \\
		y &= \sigma(z) \\
		L(y,t) &= -t\log(y) - (1-t) \log (1-y)
	\end{align}
	
	Show that if $t^i \in \{-1, +1\}$ then the minimization problem takes the following form where $w$ and $b$ are the weights parameters:
	\begin{equation}
		min_{w,b} \sum_{i=1}^{N} \log (1+\exp({-t^i(w^Tx^{(i)} + b)}))
	\end{equation}
	
	\begin{framed}
		\textbf{Solution:} \\
		We can substitute the expression for $y$, then later substitute for z:
		\begin{align}
			L(z,D) &= -t \log (\sigma(z)) - (1-t) \log (1-\sigma(z)) \\
			&= -t \log \left( \frac{1}{1+\exp (-z)}  \right) - (1-t) \log \left( 1 - \frac{1}{1+\exp (-z)}  \right) \\
			&= -t \log \left( \frac{1}{1+\exp (-z)}  \right) - (1-t) \log \left(\frac{1}{1+\exp (z)}  \right) \\
			&= t \log \left( 1 + \exp (-z) \right) + (1-t) \log \left( 1 + \exp(z) \right) \\
			L(w,b,D) &= t \log \left( 1 + \exp (-(w^Tx + b)) \right) + (1-t) \log \left( 1 + \exp(w^Tx+b) \right) \\
			&= \sum_{i=1}^{N} t^{(i)} \log \left( 1 + \exp (-(w^T x^{(i)} + b)) \right) + (1-t^{(i)}) \log \left( 1 + \exp(w^T x^{(i)}+b) \right) 
		\end{align}
		
		Thus the cost minimization problem when  $t^i \in {0, +1}$ is formulated as:
		\begin{equation}
			\min_{w,b} \sum_{i=1}^{N} t^{(i)} \log \left( 1 + \exp (-(w^T x^{(i)} + b)) \right) + (1-t^{(i)}) \log \left( 1 + \exp(w^T x^{(i)}+b) \right) 
		\end{equation}
		
		When $t^i \in \{-1, +1\}$ then we can substitute $t^{(i)} = \frac{t^{(i)} + 1}{2}$ into the expression above:
		
		\begin{equation}
			L(w,b,D) = \sum_{i=1}^{N} \frac{t^{(i)}+1}{2} \log \left( 1 + \exp (-(w^T x^{(i)} + b)) \right) + \frac{1-t^{(i)}}{2} \log \left( 1 + \exp(w^T x^{(i)}+b) \right) 
		\end{equation}
		
		When $t^{(i)} = +1$ for the $i^{th}$ training example, the second term disappears, leading to the remaining term. When $t^{(i)} = -1$ for the $i^{th}$ training example, the first term disappears, leading to the remaining term. therefore, the only difference between the two cases in the sign inside the exponential term, which has the opposite sign as $t^{(i)}$. We can simplify to the desired expression:
		
		\begin{equation}
			min_{w,b} \sum_{i=1}^{N} \log (1+\exp({-t^i(w^Tx^{(i)} + b)}))
		\end{equation}
		
	\end{framed}
	
	\section*{Question 4: Optimization}
	Assume that you are minimizing a cost function which can be written as:
	\begin{equation}
		J(w) = \frac{1}{N} \sum_{i-1}^{N} L(w, x_i, t_i)
	\end{equation}
	Where $N=1,000,000$.
	
	a) Write the one-step update rules for gradient descent (GD), stochastic GD and mini-batch SGD with batch size of 100. You can denote the gradient of the loss with respect to $w$ for each sample with $g_i = \nabla L(w,x_i,t_i)$ and your learning rate is $\eta$.
	
	b) Rank the computational cost of each iteration for GD, SGD and mini-batch SGD (with batch size of 100) from smallest to largest.
	
	\begin{framed}
		\textbf{Solution:}\\
		a) The gradient update rule for GD, SGD and mini-batch SGD are as follows:
		\begin{enumerate}
			\item{GD:}
			\begin{equation}
				w \leftarrow w - \eta \sum_{i=1}^{N} g_i
			\end{equation}
			\item{SGD:}
			\begin{equation}
				\text{Choose} \quad i \approx Uniform[1,N], w \leftarrow w - \eta g_i
			\end{equation}
			\item{mSGD:}
			\begin{equation}
				\text{Choose a subset} M \subset  {1,2, \dots , N}, w \rightarrow w - \eta \sum_{i \in M}^{|M|}g_i
			\end{equation}
		\end{enumerate}
		
		b) From smallest to largest $SGD < mSGD < GD$.\\
		SGD only requires processing of 1 training example, mSGD requires 100 batch examples and GD requires 1000000 training examples.
	\end{framed}
	
	\section*{Question 5: Optimization}
	
	Let $n \geq 1$ be an integer and let $A \in \mathbb{R}^{n \times n}$ be a symmetric matrix (non necessarily positive definite) for which all of its eigenvalues are non-zero. Let $a \in \mathbb{R}^n$ be a given vector and we consider the function $f:\mathbb{R} \rightarrow \mathbb{R}$, defined as:
	\begin{equation}
		f(x) = \frac{1}{2}(x-a)^TA^2(x-a)
	\end{equation}
	
	a) Using first and second order optimality conditions show that $f$ has a unique global minimizer on $\mathbb{R}^n$ and determine this optimizer. Denote it by $x^*$.
	
	b) Write the updates in the gradient descent algorithm with optimal step size starting from a point $x^0 \in \mathbb{R}^n$ to appropriate the optimizer $x^*$ of $f$ that has been determined in (a). Determine the step size $\alpha_k$ in each step.
	
	\begin{framed}
		\textbf{Solution:}\\
		Notice first the since $A$ is symmetric, so is $A^2$. Moreover since $A$ has non-zero eigenvalues, $A^2$ has all its eigenvalues positive, hence it is a positive definite matrix. Let us define $Q := A^2$. Observe also that the function can be rewritten as:
		
		\begin{equation}
			f(x) = \frac{1}{2}x^TQx - x^Tb + c \text{\quad Where } b := Qa, c := \frac{1}{2}a^TQa
		\end{equation}

		a) Since the optimization problem is without constraints, the first order necessary optimality condition for the minimizer reads as $\nabla f(x^*) = 0$, that is $Qx^* = b$, from where $x^* = Q^{-1}b = Q^{-1}Qa = a$. All these computations are meaningful because $Q^{-1}$ exists. The second order sufficient condition of minimality reads as $D^2f(x^*) = Q = A^2 > 0$ which is true. Hence $x^* = a$ is the unique global minimizer of $f$ on $\mathbb{R}^n$.
		
		b) The updates in the gradient descent starting from $x^0$ are:
		
		\begin{equation}
			x^{k+1} = x^{k} - \alpha_k \nabla f(x^k) = x^k - \alpha_k (Qx^k - b)
		\end{equation}
		
		Where $\alpha_k = argmin_{\alpha \in \mathbb{R}} f(x^k - \alpha \nabla f(x^k))$. 
		
		\begin{equation}
			\alpha_k = \frac{||\nabla f(x^k)||^2}{\nabla f(x^k)^T Q \nabla f(x^k)}
		\end{equation}

	\end{framed}
	
	\section*{Question 6: Mixed - Optimization and Linear Algebra}
		Let $n \geq 1$ be an integer and let $A \in \mathbb{R}^{n \times n}$ be a symmetric matrix (non necessarily positive definite) for which all of its eigenvalues are non-zero. Let $a \in \mathbb{R}^n$ be a given vector and we consider the function $f:\mathbb{R} \rightarrow \mathbb{R}$, defined as:
	\begin{equation}
		f(x) = \frac{1}{2}(x-a)^TA^2(x-a)
	\end{equation}
	
	a) Imagine that one wants to use a fixed step gradient algorithm too, to approximate $x^*$. Which is maximal range for the step size $\alpha$ in terms of the eigenvalues of $A$ that ensures global convergence for the algorithm?
	
	b) Give an example of $A \in \mathbb{R}^{2 \times 2}$ diagonal matrix that has a zero and a non-zero eigenvalue. Take $a \in \mathbb{R}^2$. Determine the global minimizers of $f$ in $\mathbb{R}^2$ in this case. What can we say about the uniqueness of them?
	
	\begin{framed}
		\textbf{Solution:}\\
		a) For the fixed step size algorithm global convergence is equivalent to $0 < \alpha < \frac{2}{\lambda_{max}(Q)}$. The maximal eigenvalue of $Q$ actually can be written in terms of the maximal (in absolute value) eigenvalue of $A$ i.e. $\lambda_{max}(Q) = \max {\lambda_i^2 : i = 1, 2, \dots , n}$, where the $\lambda_i$ are the eigenvalues of $A$ counted with multiplicity.
		
		b) An example of such a matrix is:
		
		\begin{equation}
			A = \begin{bmatrix}
				\gamma & 0 \\
				0 & 0 \\
			\end{bmatrix}
		\end{equation}
		
		Where $\gamma \neq 0$. The other option is when the elements on the main diagonal are exchanged. In this case,
		
		\begin{equation}
			Q = A^2 = \begin{bmatrix}
									\gamma^2 & 0 \\
									0 & 0 \\
								\end{bmatrix}
		\end{equation}
		
		and the function can be written as $f(x_1,x_2) = \frac{1}{2} \gamma^2 (x_1 - a_1)^2$, hence it is independent of the second variable. Setting $\nabla f(x) = 0$ one finds that the candidates for the optimizers are $x^* = (a_1,x_2)$, where $x_2 \in \mathbb{R}$ is arbitrary. Since the function is independent of the second variable and $f(a_1,x_2) = 0 \leq f(y_1,y_2)$ for any $(y_1, y_2) \in \mathbb{R}^2$, one has that all of them are global minimizers that have the same objective function value, hence they are not unique.
		
	\end{framed}
	
	\section*{Question 7: Mixed - Optimization and Linear Algebra}
	Let $n \geq 1$ be an integer and let $A \in \mathbb{R}^{n \times n}$ be a symmetric matrix (non necessarily positive definite) for which all of its eigenvalues are non-zero. Let $a \in \mathbb{R}^n$ be a given vector and we consider the function $f:\mathbb{R} \rightarrow \mathbb{R}$, defined as:

	\begin{equation}
		f(x) = \frac{1}{2}(x-a)^TA^2(x-a)
	\end{equation}
	
	a) Explain what will happen if we want to proceed with a fixed step size gradient algorithm for Question 6 (b). Does an algorithm like this converge globally? If yes, for which values of the step size $\alpha$ and to which limit point $x^*$?
	
	b) Explain what is the major difference between the cases when $A$ has at least one zero eigenvalue and when it does not, from the point of view of the gradient descent algorithms.
	
	\begin{framed}
		\textbf{Solution:}\\
		a) In the case of Question 6 (b) the problem is reduced to a 1D problem, hence a fixed step size gradient algorithm converges globally if and only if the step size $\alpha$ is in the range $0 < \alpha < \frac{2}{\gamma^2}$. From the 2D point of view what is happening is the following: choosing any initial guess $x^0 = (x^0_1, x^0_2)$, since $f$ is independent of the second variable (hence the second coordinate of its gradient is always zero), during each update in $x^{k+1} = (x^{k+1}_1,x^{k+1}_2)$ the second coordinate $x_2^{k+1}$ remains unchanged. Hence the algorithm actually converges to a global minimizer namely the one $(a_1,x_2^0)$.
		
		b) If some of the eigenvalues of $A$ are zero, $Q = A^2$ will have also the corresponding eigenvalues 0. On the other hand, since $Q$ is symmetric, it is diagonalizable, so we can see it up to a change of coordinates as a diagonal matrix with the eigenvalues on the main diagonal. As we have seen in (a), the coordinates (in the new system of coordinates, if Q was not diagonal at the first place) corresponding to the zero eigenvalues are unaffected by the gradient algorithms. And the dimension of the problem can be reduced by the number of zero eigenvalues. While for positive definite $Q$, i.e. if $A$ does not have zero eigenvalues, the problem is full dimensional. This is a major difference between the two cases.
	\end{framed}
	
	\section*{Question 8: Linear Algebra}
	If $A$ is $3 \times 3$ symmetric positive definite, then $Aq_i = \lambda_i q_i$ with positive eigenvalues and orthonormal eigenvectors $q_i$. Suppose $x = c_1q_1 + c_2q_2 + c_3q_3$.
	
	a) Compute $x^Tx$ and also $x^TAx$ in terms of the $c$ and $\lambda$.
	
	b) Looking at the ratio of $x^TAx$ in part (a) divided by $x^Tx$ in part (a), what $c$ would make that ratio as large as possible ? You can assume $\lambda_1 < \lambda_2 < .\dots < \lambda_n$. Also find $x$ where the ratio $\frac{x^TAx}{x^Tx}$ is a maximum.
	
	\begin{framed}
		\textbf{Solution:}
		a) 
		
		\begin{align}
			x^Tx &= \left(\sum_{i = 1}^{3} c_iq_i^T\right)\left(\sum_{j = 1}^{3} c_jq_j\right) \\
			&= c_1^2 q_1^T q_1 + c_1 c_2 q_1^T q_2 + \dots + c_3^2 q_3^T q_3 \\
			&= c_1^2 + c_2^2 + c_3^2 \\
			x^TAx &= \left(\sum_{i = 1}^{3} c_iq_i^T\right)\left(\sum_{j = 1}^{3} c_jAq_j\right) \\
			&= \left(\sum_{i = 1}^{3} c_iq_i^T\right)\left(\sum_{j = 1}^{3} c_j \lambda_j q_j\right) \\
			&= c_1^2 \lambda_1 q_1^T q_1 + c_1 c_2 \lambda_2 q_1^T q_2 + \dots + c_3^2 \lambda_3 q_3^T q_3 \\
			&= c_1^2 \lambda_1 + c_2^2 \lambda_2 + c_3^2 \lambda_3
		\end{align}
		
		b) We maximize $\frac{\sum_{i=1}^{3} c_i^2 \lambda_i}{\sum_{i=1}^{3} c_i^2}$ when $c_1 = c_2 = 0$ so $x = c_3q_3$ is a multiple of the eigenvector $q_3$ with the largest eigenvalue $\lambda_3$.
	\end{framed}
	
	\section*{Question 9: Linear Regression}
	a) Let $f$ be some function so that $f(\theta_0,\theta_1)$ outputs a number. For this problem, $f$ is some arbitrary/unknown smooth function (not necessarily the cost function of linear regression, so $f$ may have local optima). Suppose we use gradient descent to try to minimize $f(\theta_0,\theta_1)$ as a function of $\theta_0$ and $\theta_1$. Which of the following statements are true and which are false? Explain.
	\begin{enumerate}
		\item Even if the learning rate $\alpha$ is very large, every iteration of gradient descent will decrease the value of $f(\theta_0,\theta_1)$.
		\item If the learning rate is too small, then gradient descent may take a very long time to converge.
		\item If $\theta_0$ and $\theta_1$ are initialized at a local minimum, then one iteration will not change their values.
		\item If $\theta_0$ and $\theta_1$ are initialized so that $\theta_0=\theta_1$, then by symmetry (because we do simultaneous updates to the two parameters), after one iteration of gradient descent, we will still have $\theta_0=\theta_1$.
		\item If the first few iterations of gradient descent cause $f(\theta_0,\theta_1)$ to increase rather than decrease, then the most likely cause is that we have set the learning rate to too large.
		\item No matter how $\theta_0$ and $\theta_1$ are initialized, so long as learning rate is sufficiently small, we can safely expect gradient descent to converge to the same solution.
		\item Setting the learning rate to be very small is not harmful, and can only speed up the convergence of gradient descent.
	\end{enumerate}
	
	b) Suppose that for some linear regression problem, we have some training set, and for our training set we managed to find some $\theta_0, \theta_1$ such that cost function $J(\theta_0,\theta_1)=0$. Which of the statements below are true and which are false? Explain.
	\begin{enumerate}
		\item For this to be true, we must have $y^{(i)}=0$ for every value of $i=1,2,\dots,m$.
		\item Gradient descent is likely to get stuck at a local minimum and fail to find the global minimum.
		\item For this to be true, we must have $\theta_0=0$ and $\theta_1=0$ so that $\theta_0 + \theta_1x=0$.
		\item Our training set can be fit perfectly by a straight line, i.e., all of our training examples lie perfectly on some straight line.
	\end{enumerate}
	
	\begin{framed}
		\textbf{Solution:}\\
		a)
		\begin{enumerate}
			\item (True) If the learning rate is small, gradient descent ends up taking an extremely small step on each iteration, and therefor can take a long time to converge.
			\item (True) At a local minimum, the derivative (gradient) is zero, so gradient descent will not change the parameters.
			\item (False) If the learning rate is too large, one step of gradient descent can actually vastly "overshoot" and actually increase the value of $f(\theta_0,\theta_1)$.
			\item (False) The updates to $\theta_0$ and $\theta_1$ are different (even though we're doing simultaneous updates), so there's no particular reason to update them to be same after one iteration of gradient descent.
			\item (True) if alpha were small enough, then gradient descent should always successfully take a tiny small downhill and decrease $f(\theta_0,\theta_1)$ at least a little bit. If gradient descent instead increases the objective value, that means alpha is too large (or you have a bug in your code!).
			\item (False) This is not true, depending on the initial condition, gradient descent may end up at different local optima.
			\item (False) If the learning rate is small, gradient descent ends up taking an extremely small step on each iteration, so this would actually slow down (rather than speed up) the convergence of the algorithm.
		\end{enumerate}
		
		b)
		\begin{enumerate}
			\item (False) So long as all of our training examples lie on a straight line, we will be able to find $\theta_0$ and $\theta_1$ so that $J(\theta_0,\theta_1)=0$. It is not necessary that $y^{(i)}$ for all our examples.
			\item (False) Not possible as the objective is convex.
			\item (False) If $J(\theta_0,\theta_1)=0$ that means the line defined by the equation $y = \theta_0 + \theta_1x$ perfectly fits all of our data. There's no particular reason to expect that the values of $\theta_0$ and $\theta_1$ that achieve this are both 0 (unless $y^{(i)}=0$ for all of our training examples).
			\item (True) Obvious since all the examples lie on straight line.
		\end{enumerate} 
	\end{framed}
	
	\section*{Question 10: Logistic Regression}
	You are training a classification model with logistic regression. Which of the following statements are true and which are false? Explain.
	\begin{enumerate}
		\item Adding many new features to the model helps prevent overfitting on the training set.
		\item Introducing regularization to the model always results in equal or better performance on examples not in the training set.
		\item Introducing regularization to the model always results in equal or better performance on the training set.
		\item Adding a new feature to the model always results in equal or better performance on the training set
		\item Using a very large value regularization parameter cannot hurt the performance of your hypothesis; the only reason we do not set to be too large is to avoid numerical problems.
		\item Because regularization causes cost function to no longer be convex, gradient descent may not always converge to the global minimum.
		\item Using too large a value of regularization parameter can cause your hypothesis to underfit the data.
		\item Because logistic regression outputs values $0 \leq y \leq 1$, its range of output values can only be "shrunk" slightly by regularization anyway, so regularization is generally not helpful for it.
	\end{enumerate}
	
	\begin{framed}
		\textbf{Solution:}\\
		\begin{enumerate}
			\item (False) Adding many new features gives us more expressive models which are able to better fit our training set. If too many new features are added, this can lead to overfitting of the training set.
			\item (False) If we introduce too much regularization, we can underfit the training set and this can lead to worse performance even for examples not in the training set.
			\item (False) If we introduce too much regularization, we can underfit the training set and have worse performance on the training set.
			\item (True) Adding many new features gives us more expressive models which are able to better fit our training set. If too many new features are added, this can lead to overfitting of the training set.
			\item (False) Using a very large value of regularization parameter can lead to underfitting of the training set.
			\item (False) Regularized logistic regression and regularized linear regression are both convex, and thus gradient descent will still converge to the global minimum.
			\item (True) A large value of results in a large regularization parameter regularization penalty and thus a strong preference for simpler models which can underfit the data.
			\item (False) No it is needed as it prevents overfitting
		\end{enumerate}
	\end{framed}
	
	
	%%%%%%%%%%% end of doc
\end{document}